\documentclass{ltjsarticle}
\begin{document}

\title{はじめての\TeX }
\author{Yuya TANO}
\maketitle
\section{はじめての\TeX がLua\TeX なんて粋だね}
%\section{はじめての\TeXがLua\TeXなんて粋だね}

%こうやって文字を打ちます。
\subsection{小見出し!}
あたり前過ぎて気に止めることもないですが、きとんと改行命令を出していなくても自動で改行します。
\\あ、ギリ改行されなかったか。

\begin{table}[h]
	\caption{光速度の測定の歴史}
	\label{table:SpeedOfLight}
	\centering
	 \begin{tabular}{clll}
	  \hline
	  西暦 & 測定者 & 測定方法 & 測定結果 \\
	   & & & $\times 10^8$ [m/sec] \\
	  \hline \hline
	  1638 & Galileo & 二人が離れてランプの光を見る & (音速10倍以上) \\
	  1675 & Roemer & 木星の衛星の観測から & 2 \\
	  1728 & Bradley & 星の収差から & 3.01 \\
	  1849 & Fizeau & 高速に回転する歯車を通過する光を見る & 3.133 \\
	  1862 & Foucault & 高速に回転する鏡の光の角度変化 & 2.99796 \\
	  今日 & (定義) & & 2.99792458 \\
	  \hline
	 \end{tabular}
\end{table}
\end{document}